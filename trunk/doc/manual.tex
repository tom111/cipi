\documentclass[12pt]{amsart}

\usepackage{latexsym}
\usepackage{amscd}
\usepackage{epsf}
\usepackage{color}
\usepackage{amsopn,fixltx2e,mparhack,microtype}
\usepackage{amsfonts}
\usepackage{euscript,mathrsfs} 
\usepackage{graphicx}    % standard LaTeX graphics tool
\usepackage{a4}

\newcommand{\set}[1]{\left\lbrace #1 \right\rbrace} % standard set
\newcommand{\defas}{\mathrel{\mathop{:}}=}   % Definition
\renewcommand{\subset}{\subseteq}  % Subsetsymbols
\renewcommand{\supset}{\supseteq}
\newcommand{\todo}[1]{\{ \huge{Todo:}\normalsize #1 \}} % Todo
\DeclareMathOperator*{\bigtimes}{\textnormal{\Large $\times$}} % Cartesian Product
\newcommand{\comm}[1]{ }
\providecommand{\abs}[1]{\left\lvert#1\right\rvert} 
\providecommand{\norm}[1]{\left\lVert#1\right\rVert}
\newcommand{\E}{\mathbb{E}} % expectation
\renewcommand{\P}{\mathbb{P}} % probability
\renewcommand{\d}{\mathrm{d}\:\!} %differential 
\newcommand{\e}{\mathrm{e}} % Eulers Number
\renewcommand{\i}{\mathrm{i}} % Imaginary Unit
\newcommand{\Ind}[1]{\mathbbm{1}_{\lbrace #1 \rbrace}} % indicator function
\newcommand{\Id}{\mathbbm{1}} % The identity map
\newcommand{\ie}{i.e.\;}  % make i.e. not the end of a sentence
\newcommand{\eg}{e.g.\;}  % make e.g. not the end of a sentence
\newcommand{\etc}{etc.\;}
\newcommand{\iid}{i.i.d.\;} % make i.i.d. not the end of sentence
\newcommand{\cipi}{\texttt{CIPI}\,}

\theoremstyle{plain}% default
\newtheorem{thm}{Theorem}
\newtheorem{lem}[thm]{Lemma}
\newtheorem{prop}[thm]{Proposition}
\newtheorem*{cor}{Corollary}

\theoremstyle{definition}
\newtheorem{defn}[thm]{Definition}
\newtheorem{conj}[thm]{Conjecture}
\newtheorem{exmp}[thm]{Example}

\theoremstyle{remark}
\newtheorem*{rem}{Remark}
\newtheorem*{note}{Note}
\newtheorem{case}{Case}
%%%%%%%%%%%%%%%%%%%%%%%%%%%%%%%%%%%%%%%%%%%%%%%%%%%%%%%%%%%%%%%%%%%%%

\begin{document}

\title[CIPI]{Computing Information Projections Iteratively with \cipi}

\author{L. Steiner and T. Kahle}
\address{Max Planck Institute for Mathematics in the Sciences, Inselstrasse 22, D-04103 Leipzig, Germany}
\email{\{kahle,steiner\}@mis.mpg.de}

\date{\today}
\begin{abstract}
  some abstract\ldots
\end{abstract}

%% Maketitle here would insert a pagebreak
\maketitle

\section{Introduction}
In this paper we describe the software tool \cipi. It can be used to compute so
called information projections. Given an exponential family $\mathcal{E}$ of probability
measures associated to a complete hypergraph on $\set{1,\ldots,N}$ (a ``hierarchical model'')
\cipi computes, for a given measure $\hat{P}$, the minimizer of 
\begin{equation*}
  D(\hat{P}\parallel Q) \qquad Q\in\mathcal{E}, 
\end{equation*} where $D$ is the \emph{Kullback-Leibler Divergence}. These
Minimizers are of high intereset in several applications.
The paper is organized as follows. In Section \ref{sec:theory} we introduce the
theoretical background. In Section \ref{sec:usage} the technical details of
using \cipi are described. 

\begin{exmp}
  Consider the problem of contingency tables. Assume, in a survey 1000 people in
  Country A and Country B have been asked bout their smoking behaviour. This
  yielded the following results
  \begin{center}
  \begin{tabular}[]{|l|c|c|}
    \hline
    & male & female \\
    \hline
    smoking & 152 & 271 \\
    non-smoking & 319 & 258\\
    \hline
  \end{tabular}\hspace{1cm}
  \begin{tabular}[]{|l|c|c|}
    \hline
    & male & female \\
    \hline
    smoking & 1 & 2 \\
    non-smoking & 3 & 4\\
    \hline
  \end{tabular}
\end{center}

  A statistician might come up with the idea to test for independence by
  measuring the Kullback-Leibler Divergence of the empirical distribution to the
  ``closest'' approximation within the set of indipendent distributions, \ie the
  set 
  \begin{equation*}
    \mathcal{E}_1 \defas \set{P : P\left( x_1,x_2 \right)) = P_1(x_1)P_2(x_2)}
  \end{equation*}

  He would now create two files called \texttt{A.dat}, \textttT{B.dat} with
  containing, one sample per line. So \texttt{A.dat} would contain 152 lines
  with the content ``00''. 271 lines with ``01'', 319 lines with ``10'' and 258
  lines containing ``11''. \cipi will build the emperical measure from the data.
  
\end{exmp}

\section{Information Theory}
\label{sec:theory}
In this section we introduce exponential families of probability
measures, the Kullback-Leibler divergence and other notions.
Throughout the whole paper, the set of states of the system is of compositional
structure. By that we mean, for each site $v$ in a finite set $V \defas
\set{1,\ldots, N}$ we have a configuration space $\mathcal{X}_v$. The set of all
possible configurations is given as 
\begin{equation*}
\mathcal{X}_V \defas \bigtimes_{v\in V}\mathcal{X}_v,
\end{equation*}and likewise for each subset $A\subset V$, $\mathcal{X}_A \defas
\bigtimes_{v\in A}\mathcal{X}_v$.

\subsection{Families of Probability Measures} 
We now study probability distributions on the set
$\mathcal{X}_V$. Consider 
\[
\overline{\mathcal{P}}(\mathcal{X}_V)
\defas \set{P \in \mathbb{R}^{\mathcal{X}_V} : P(x) \geq 0,\,
  \sum_{x\in\mathcal{X}_V} P(x) = 1},
\]  
the probability measures on
the set $\mathcal{X}_V$.
$\overline{\mathcal{P}}(\mathcal{X}_V)$ is naturally embedded in
\[
\mathbb{R}^{\mathcal{X}_V} \defas \set{f : \mathcal{X}_V \to \mathbb{R}},
\] the vector space of real valued functions on $\mathcal{X}_V$.
$\overline{\mathcal{P}}(\mathcal{X}_V)$ has the geometrical structure of a
$(|\mathcal{X}_V| -1)$ dimensional simplex. For $P\in
\overline{\mathcal{P}}(\mathcal{X}_V)$, we call \[ \mbox{supp}(P) \defas
\set{x \in \mathcal{X}_V : P(x) > 0} \] the \emph{support} of
$P$. For two distributions $P,Q \in \overline{\mathcal{P}}(\mathcal{X}_V)$,
we can now define a notion of distance by
\begin{equation*}
  D(P\parallel Q ) \defas \left\lbrace 
  \begin{array}{ll}
    \sum_{x\in\mathcal{X}} P(x) \log_2
    \frac{P(x)}{Q(x)} & \mbox{ if } \mbox{supp}(P) \subset
    \mbox{supp}(Q) \\ 
    \infty & \mbox{ otherwise}.
  \end{array} \right.
\end{equation*}
$D(P\parallel Q)$ is called the \emph{Kullback-Leibler divergence} or
\emph{relative entropy}, although not a metric it is non-negative and
equals zero if and only if $P=Q$. 

In Information Geometry one studies families of probability
measures i.e. submanifolds of $\overline{\mathcal{P}}(\mathcal{X}_V)$. A
very natural class of such families arises if we consider the
exponential map
\begin{equation*}
  \exp : \mathbb{R}^{\mathcal{X}_V} \to \overline{\mathcal{P}}(\mathcal{X}_V) 
  \qquad f \mapsto \frac{\e^{(f)}}{\sum_{x\in\mathcal{X}_V} \e^{(f(x))}}.
\end{equation*}
Then an \emph{exponential family} $\mathcal{E}_\mathcal{I}$ is the
image of a linear subspace $\mathcal{I}$ of $\mathbb{R}^{\mathcal{X}_V}$.
Exponential families are well known in statistical science. It can
be seen, that minimization of $D(P\parallel Q)$ with respect to $Q$
in an exponential family $\mathcal{E}$, is in fact equivalent to maximum
likelihood estimation for $P$ in the family $\mathcal{E}$. On the
other hand, geometrically, minimization of $D(P\parallel Q)$ could be
interpreted as a projection, since $D$ is a kind of distance. These projections 
play an important role in information geometry. 


\subsection{Interaction Spaces}
We exploit the compositional structure of $\mathcal{X}_V$ to 
define the notion of interactions, that are localized in a subset of the nodes. 

For any given $A\subset V$, we
write $x\in \mathcal{X}_V$ as $x = (x_A,x_{V\setminus
  A})$ with $x_A\in \mathcal{X}_A$, $x_{V \setminus
  A}\in\mathcal{X}_{V \setminus A}$, \ie we distinguish between components in
  $A$ and outside $A$. Then define ${\mathcal I}_A$ to be the
subspace of functions that do not depend on the configurations
$x_{V \setminus A}$:
\begin{align*}
  {\mathcal I}_A \defas \left\{ f \in {\mathbb R}^\mathcal{X} : 
    f(x_A,x_{V \setminus A}) = f(x_A, x_{V
      \setminus A}') \mbox{ for all $x_A \in \mathcal{X}_A$, $x_{V \setminus A}, 
      x_{V \setminus A}' \in \mathcal{X}_{V
        \setminus A}$} \right\}.
\end{align*}
Using these spaces as building blocks, one can define the interaction
space corresponding to a hypergraph $\mathcal{A}$ 
\[
{\mathcal I}_\mathcal{A} \defas \sum_{ A \in \mathcal{A} } {\mathcal I}_A.
\]
The sum is the inner sum in $\mathbb{R}^{\mathcal{X}_V}$, \ie just the span of all
the vectors on the right hand side.
Associated with each of the interaction spaces is an exponential family
$\mathcal{E}_{\mathcal{A}}
\defas \exp(\mathcal{I}_k)$. For two hypergraphs $\mathcal{A} \subset
\mathcal{B}$ one hase $\mathcal{E}_{\mathcal{A}}\subset
\mathcal{E}_{\mathcal{B}}$. Typically one studies hierarchies of hypergraphs and
exponential families\cite{KahOlbJosAy08}. The main example beeing the
hypergraphs of fixed interaction order 
\begin{equation*}
  \mathcal{A}_k \defas \set{A\subset V : \abs{A} \leq k}  
\end{equation*}
This gives us a hierarchy of exponential families: 
\[ {\mathcal E}_{1} \; \subseteq \;
{\mathcal E}_{2} \subseteq \; \dots \subseteq \; {\mathcal E}_{N}.
\] 
It was studied in \cite{Amari01,ayknauf06} and found various
applications in the theory of neural networks. 

Note that $\mathcal{E}_N$ is the relative interior of
$\overline{\mathcal{P}}(\mathcal{X}_V)$. In general, in the image of the
exponential map, one has only distributions with full support. Probability
zero corresponds to infinite energy which is only achievable by 
limits of sequences of probability measures in an exponential family. It is an
open question which of the possible support sets are achievable by such limit
probability measures in a given exponential family. This questions is connected to
the face structure of a convex polytope, the so called marginal polytope. See
\cite{CziMa02,kahleay06} for details.
For our application one can just pass to the closure(in $\mathbb{R}^n$) 
of the exponential family. The algorithm we use, the iterated projection
algorithm \cite{csiszarshields04}, is specially suited to scope
with the problem of unobserved configurations. 


\section{Usage}
\label{sec:usage}
\subsection{Parameters}
Every call of \cipi is must be followed by the name of a file containing
parameters. This file allows the following options
\begin{description}
  \item[N] The number of nodes.
  \item[SetIterations] The number of times every set of cardinality $k$ is
    visited in the projection loop.
  \item[SlidingWindow] Specifies how the data is read. If SlidingWindow is set
    to \textsc{true} then a datapoint can be a long sequence of symbols and the
    statistic is generated using a sliding window of lenght N
  \item[SamplesBinary] ???
  \item[Aplhabet] specifies the alphabet which is used, \eg ``01'' or
    ``AGCT''. 
  \item[Hpergraph] Describes the Hypergraph of the hierarchical model
\end{description}

\subsection{Output}
Currently \cipi prints to sdtout.

\bibliographystyle{amsalpha}
\bibliography{alotofstuff}

\end{document}





